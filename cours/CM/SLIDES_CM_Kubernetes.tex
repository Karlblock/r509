\documentclass[aspectratio=169]{beamer}

% ================================
% PACKAGES
% ================================
\usepackage[utf8]{inputenc}
\usepackage[T1]{fontenc}
\usepackage[french]{babel}
\usepackage{lmodern}
\usepackage{graphicx}
\usepackage{tikz}
\usetikzlibrary{shapes.geometric, arrows, positioning, fit, backgrounds}
\usepackage{listings}
\usepackage{xcolor}
\usepackage{hyperref}
\usepackage{fontawesome5}

% ================================
% COULEURS PERSONNALISÉES
% ================================
\definecolor{iutblue}{RGB}{41,128,185}
\definecolor{iutcyan}{RGB}{52,152,219}
\definecolor{iutgreen}{RGB}{39,174,96}
\definecolor{iutorange}{RGB}{230,126,34}
\definecolor{iutred}{RGB}{231,76,60}
\definecolor{iutgray}{RGB}{149,165,166}
\definecolor{iutdarkgray}{RGB}{52,73,94}
\definecolor{codebg}{RGB}{248,248,248}

% ================================
% THÈME BEAMER
% ================================
\usetheme{Madrid}
\usecolortheme{default}

\setbeamercolor{palette primary}{bg=iutblue,fg=white}
\setbeamercolor{palette secondary}{bg=iutcyan,fg=white}
\setbeamercolor{palette tertiary}{bg=iutdarkgray,fg=white}
\setbeamercolor{palette quaternary}{bg=iutblue,fg=white}
\setbeamercolor{structure}{fg=iutblue}
\setbeamercolor{section in toc}{fg=iutblue}
\setbeamercolor{subsection in toc}{fg=iutcyan}
\setbeamercolor{frametitle}{bg=iutblue,fg=white}
\setbeamercolor{title}{fg=iutblue}
\setbeamercolor{block title}{bg=iutblue,fg=white}
\setbeamercolor{block body}{bg=iutblue!10,fg=black}
\setbeamercolor{block title alerted}{bg=iutorange,fg=white}
\setbeamercolor{block body alerted}{bg=iutorange!10,fg=black}
\setbeamercolor{block title example}{bg=iutgreen,fg=white}
\setbeamercolor{block body example}{bg=iutgreen!10,fg=black}

\setbeamertemplate{navigation symbols}{}
\setbeamertemplate{footline}[frame number]
\setbeamertemplate{itemize items}[circle]
\setbeamertemplate{enumerate items}[default]

% ================================
% BLOCS PERSONNALISÉS
% ================================
\newenvironment<>{important}[1][Important]{%
  \begin{alertblock}#2{#1}}{\end{alertblock}}

\newenvironment<>{exemple}[1][Exemple]{%
  \begin{exampleblock}#2{#1}}{\end{exampleblock}}

\newenvironment<>{objectif}[1][Objectifs]{%
  \setbeamercolor{block title}{bg=iutcyan,fg=white}
  \setbeamercolor{block body}{bg=iutcyan!10,fg=black}
  \begin{block}#2{#1}}{\end{block}}

\newenvironment<>{reference}[1][Référence]{%
  \setbeamercolor{block title}{bg=iutgray,fg=white}
  \setbeamercolor{block body}{bg=iutgray!10,fg=black}
  \begin{block}#2{#1}}{\end{block}}

% ================================
% LISTINGS
% ================================
\lstdefinelanguage{yaml}{
    keywords={apiVersion, kind, metadata, name, namespace, labels, annotations, spec, selector, ports, protocol, port, targetPort, type, containers, image, volumeMounts, mountPath, readOnly, volumes, configMap, secret, persistentVolumeClaim, claimName, emptyDir, livenessProbe, readinessProbe, httpGet, path, initialDelaySeconds, periodSeconds, rules, host, http, paths, backend, service, resources, limits, requests, env, envFrom},
    sensitive=true,
    comment=[l]{\#},
    morestring=[b]',
    morestring=[b]"
}

\lstset{
    basicstyle=\ttfamily\tiny,
    backgroundcolor=\color{codebg},
    frame=single,
    breaklines=true,
    keywordstyle=\color{iutblue}\bfseries,
    commentstyle=\color{iutgreen}\itshape,
    stringstyle=\color{iutorange},
    showstringspaces=false,
    tabsize=2,
    numbers=none
}

% ================================
% MÉTADONNÉES
% ================================
\title{Virtualisation Microservice Avancée}
\subtitle{Kubernetes : Labels, Annotations, Ingress et Volumes}
\author{Charles SIEPEN}
\institute{IUT Grand Ouest Normandie \\ BUT Informatique - Semestre 5 \\ Module R5.09 - Virtualisation Avancée}
\date{16 novembre 2025}

% ================================
% DÉBUT DU DOCUMENT
% ================================
\begin{document}

% ================================
% PAGE DE TITRE
% ================================
\begin{frame}
    \titlepage
    \vspace{-0.5cm}
    \begin{center}
        \small\textcolor{iutgray}{Contenu basé sur le cours Linux Foundation - Introduction to Kubernetes}
    \end{center}
\end{frame}

% ================================
% PLAN
% ================================
\begin{frame}{Plan}
    \tableofcontents
\end{frame}

% ================================
% INTRODUCTION
% ================================
\section{Introduction}

\begin{frame}{Contexte}
    \begin{objectif}
        Ce CM consolide vos connaissances TD/TP sur 3 axes essentiels :
        \begin{enumerate}
            \item \textbf{Labels et Annotations} : Différence et nomenclature
            \item \textbf{Ingress, Services et Pods} : Flux réseau complet
            \item \textbf{Volumes} : Persistance vs non-persistance
        \end{enumerate}
    \end{objectif}

    \vspace{0.3cm}

    \begin{important}[Point clé]
        Ces concepts sont \textbf{fondamentaux} et évalués au CC.
    \end{important}
\end{frame}

% ================================
% LABELS ET ANNOTATIONS
% ================================
\section{Labels et Annotations}

\begin{frame}[plain,c]
    \centering
    \Huge\textcolor{iutblue}{Partie 1}

    \vspace{0.5cm}

    \huge Labels et Annotations

    \vspace{1cm}

    \large\textcolor{iutgray}{Organisation et métadonnées des ressources}
\end{frame}

\begin{frame}{Règle d'or}
    \begin{center}
        \Large
        \textcolor{iutblue}{\textbf{Labels}} $\rightarrow$ Sélection et comportement K8s

        \vspace{0.5cm}

        \textcolor{iutorange}{\textbf{Annotations}} $\rightarrow$ Métadonnées (sans sélection)
    \end{center}

    \vspace{1cm}

    \begin{block}{Impact}
        \begin{itemize}
            \item \textbf{Labels} : Utilisés par les \texttt{selectors} (Services, Deployments...)
            \item \textbf{Annotations} : Informations techniques pour outils externes
        \end{itemize}
    \end{block}
\end{frame}

\begin{frame}{Labels - Caractéristiques}
    \textbf{Rôle :}
    \begin{itemize}
        \item Sélectionner des ensembles d'objets via \texttt{selectors}
        \item Organiser et regrouper les ressources
        \item Influencer le comportement de Kubernetes
    \end{itemize}

    \vspace{0.3cm}

    \textbf{Contraintes :}
    \begin{itemize}
        \item Clé : max 63 caractères
        \item Valeur : max 63 caractères
        \item Format : alphanumérique, tirets, underscores, points
    \end{itemize}
\end{frame}

\begin{frame}[fragile]{Labels - Exemple pratique}
\begin{lstlisting}[language=yaml]
apiVersion: v1
kind: Service
metadata:
  name: webapp-service
spec:
  selector:
    app: webapp        # Selectionne les pods avec ce label
    tier: frontend     # ET ce label
  ports:
    - protocol: TCP
      port: 80
      targetPort: 8080
\end{lstlisting}

    \vspace{0.3cm}

    \textbf{Résultat :} Le Service route le trafic vers tous les pods ayant \texttt{app: webapp} \textbf{ET} \texttt{tier: frontend}.
\end{frame}

\begin{frame}{Labels - Nomenclature recommandée}
    \begin{important}[Attention]
        Une nomenclature incohérente peut causer :
        \begin{itemize}
            \item Collisions entre sélecteurs
            \item Règles ambiguës (NetworkPolicy)
            \item Difficultés de maintenance
        \end{itemize}
    \end{important}

    \vspace{0.3cm}

    \textbf{Labels Kubernetes standard :}
    \begin{itemize}
        \item \texttt{app.kubernetes.io/name} : nom de l'application
        \item \texttt{app.kubernetes.io/instance} : instance spécifique
        \item \texttt{app.kubernetes.io/version} : version
        \item \texttt{app.kubernetes.io/component} : rôle (database, frontend...)
    \end{itemize}
\end{frame}

\begin{frame}{Labels et Selectors - Illustration}
    \begin{center}
        \includegraphics[height=0.75\textheight]{../../images/Labels2023.png}
    \end{center}
\end{frame}

\begin{frame}{Groupement de Pods avec Labels et Selectors}
    \begin{center}
        \includegraphics[height=0.75\textheight]{../../images/Grouping_of_Pods_using_Labels_and_Selectors.png}
    \end{center}
\end{frame}

\begin{frame}{Selectors en action}
    \begin{center}
        \includegraphics[height=0.75\textheight]{../../images/Selectors2023.png}
    \end{center}
\end{frame}

\begin{frame}{Annotations - Caractéristiques}
    \textbf{Rôle :}
    \begin{itemize}
        \item Stocker des informations techniques
        \item Configurer des outils externes (monitoring, CI/CD)
        \item Documenter les ressources
        \item Transmettre des données aux contrôleurs
    \end{itemize}

    \vspace{0.3cm}

    \textbf{Différences avec les labels :}
    \begin{itemize}
        \item Aucune limite de taille stricte
        \item Peuvent contenir JSON/YAML
        \item \textcolor{iutred}{\textbf{Ne participent PAS à la sélection}}
    \end{itemize}
\end{frame}

\begin{frame}[fragile]{Annotations - Exemple Ingress}
\begin{lstlisting}[language=yaml]
apiVersion: networking.k8s.io/v1
kind: Ingress
metadata:
  name: webapp-ingress
  annotations:
    # Configuration pour Ingress Controller
    nginx.ingress.kubernetes.io/rewrite-target: /
    nginx.ingress.kubernetes.io/ssl-redirect: "true"
    nginx.ingress.kubernetes.io/rate-limit: "100"

    # Metadonnees pour monitoring
    prometheus.io/scrape: "true"
    prometheus.io/port: "9090"
\end{lstlisting}
\end{frame}

\begin{frame}{Labels vs Annotations - Récapitulatif}
    \begin{center}
        \begin{tabular}{|l|c|c|}
            \hline
            \textbf{Critère} & \textbf{Labels} & \textbf{Annotations} \\ \hline
            Sélection ? & \textcolor{iutgreen}{\textbf{OUI}} & \textcolor{iutred}{\textbf{NON}} \\ \hline
            Impact K8s ? & Oui & Non \\ \hline
            Taille & Limitée & Flexible \\ \hline
            Exemple & \texttt{app: frontend} & \texttt{description: ...} \\ \hline
        \end{tabular}
    \end{center}
\end{frame}

% ================================
% INGRESS / SERVICES / PODS
% ================================
\section{Ingress, Services et Pods}

\begin{frame}[plain,c]
    \centering
    \Huge\textcolor{iutblue}{Partie 2}

    \vspace{0.5cm}

    \huge Ingress, Services et Pods

    \vspace{1cm}

    \large\textcolor{iutgray}{Flux réseau et exposition des applications}
\end{frame}

\begin{frame}{Flux réseau - Vue d'ensemble}
    \begin{important}[Chaîne du trafic]
        \centering
        Client externe $\rightarrow$ Ingress Controller $\rightarrow$ Ingress (règles) $\rightarrow$ Service $\rightarrow$ Pods
    \end{important}

    \vspace{1cm}

    \begin{block}{Point clé}
        Chaque composant a un rôle distinct dans l'acheminement du trafic.
    \end{block}
\end{frame}

\begin{frame}{Distinction CRITIQUE}
    \begin{columns}
        \begin{column}{0.5\textwidth}
            \textbf{Ingress (objet)}
            \begin{itemize}
                \item \textcolor{iutblue}{Ressource Kubernetes}
                \item Décrit les règles de routage
                \item Fichier YAML déclaratif
                \item \textcolor{iutred}{\textbf{Ne fait rien seul}}
            \end{itemize}
        \end{column}

        \begin{column}{0.5\textwidth}
            \textbf{Ingress Controller}
            \begin{itemize}
                \item \textcolor{iutgreen}{Composant logiciel (pod)}
                \item Reverse proxy (nginx, Traefik...)
                \item Implémente les règles
                \item \textcolor{iutgreen}{\textbf{Gère le trafic réel}}
            \end{itemize}
        \end{column}
    \end{columns}

    \vspace{0.5cm}

    \begin{important}[Erreur fréquente]
        Créer un Ingress sans déployer d'Ingress Controller = \textcolor{iutred}{\textbf{inutile !}}
    \end{important}
\end{frame}

\begin{frame}{Contrôleurs Ingress populaires}
    \begin{itemize}
        \item \textbf{ingress-nginx} : Basé sur Nginx (le plus utilisé)
        \item \textbf{Traefik} : Moderne avec dashboard
        \item \textbf{HAProxy Ingress} : Performant pour charge élevée
        \item \textbf{Contour} : Basé sur Envoy proxy
    \end{itemize}

    \vspace{0.5cm}

    \textbf{Vérification :}
    \begin{example}[Commande]
        \texttt{kubectl get pods -n ingress-nginx}

        Doit afficher le pod \texttt{ingress-nginx-controller}
    \end{example}
\end{frame}

\begin{frame}{Flux complet illustré}
    \begin{center}
        \begin{tikzpicture}[
            node distance=1.5cm,
            box/.style={rectangle, draw, fill=blue!10, text width=2cm, text centered, minimum height=0.8cm, font=\tiny},
            arrow/.style={->, >=stealth, thick}
        ]
            \node[box, fill=red!20] (client) {Client};
            \node[box, fill=orange!20, right=of client] (ctrl) {Ingress\\Controller};
            \node[box, fill=cyan!20, below=0.5cm of ctrl] (ing) {Ingress\\(règles)};
            \node[box, fill=green!20, right=of ctrl] (svc) {Service};
            \node[box, fill=blue!20, right=of svc] (pod1) {Pod 1};
            \node[box, fill=blue!20, above=0.3cm of pod1] (pod2) {Pod 2};
            \node[box, fill=blue!20, below=0.3cm of pod1] (pod3) {Pod 3};

            \draw[arrow] (client) -- node[above, font=\tiny] {HTTP} (ctrl);
            \draw[arrow, dashed] (ctrl) -- node[left, font=\tiny] {lit} (ing);
            \draw[arrow] (ctrl) -- node[above, font=\tiny] {forward} (svc);
            \draw[arrow] (svc) -- (pod1);
            \draw[arrow] (svc) -- (pod2);
            \draw[arrow] (svc) -- (pod3);
        \end{tikzpicture}
    \end{center}

    \textbf{Scénario :} Utilisateur accède à \texttt{http://app.iut.local/api}
\end{frame}

\begin{frame}{Services - Rôle et types}
    \textbf{Abstraction pour cibler des pods via labels}

    \vspace{0.3cm}

    \begin{itemize}
        \item \textbf{ClusterIP} (défaut) : IP interne au cluster
        \item \textbf{NodePort} : Expose un port sur chaque nœud
        \item \textbf{LoadBalancer} : Provisionne un LB externe (cloud)
    \end{itemize}

    \vspace{0.3cm}

    \textbf{Fonctionnement :}
    \begin{enumerate}
        \item Service sélectionne pods via labels
        \item Kubernetes crée des Endpoints (liste IPs)
        \item Trafic réparti (load balancing round-robin)
    \end{enumerate}
\end{frame}

\begin{frame}{Accès via IP des Pods (problématique)}
    \begin{center}
        \includegraphics[height=0.65\textheight]{../../images/1.A_Scenario_Where_a_User_Is_Accessing_Pods_via_their_IP_Addresses.png}
    \end{center}
    \begin{important}[Problème]
        Les IPs des pods changent à chaque recréation !
    \end{important}
\end{frame}

\begin{frame}{Nouveau Pod après crash}
    \begin{center}
        \includegraphics[height=0.75\textheight]{../../images/2.A_New_Pod_Is_Created_After_an_Old_One_Terminated_Unexpectedly.png}
    \end{center}
\end{frame}

\begin{frame}{Solution : Service Object}
    \begin{center}
        \includegraphics[height=0.65\textheight]{../../images/Accessing_the_Pods_Using_Service_Object.png}
    \end{center}
    \begin{block}{Avantage}
        Le Service maintient une IP stable et route vers les pods disponibles.
    \end{block}
\end{frame}

\begin{frame}{kube-proxy, Services et Endpoints}
    \begin{center}
        \includegraphics[width=0.85\textwidth]{../../images/asset-v1_LinuxFoundationX+LFS158x+1T2022+type@asset+block@kube-proxy__Services__and_Endpoints.png}
    \end{center}
\end{frame}

\begin{frame}{NodePort - Exposition externe}
    \begin{center}
        \includegraphics[width=0.8\textwidth]{../../images/NodePort2023.png}
    \end{center}
\end{frame}

\begin{frame}{LoadBalancer - Cloud Provider}
    \begin{center}
        \includegraphics[width=0.8\textwidth]{../../images/LoadBalancer2023.png}
    \end{center}
\end{frame}

\begin{frame}{ExternalIP}
    \begin{center}
        \includegraphics[width=0.75\textwidth]{../../images/Ch_10_-_9_ExternalIP.png}
    \end{center}
\end{frame}

\begin{frame}{NetworkPolicy - Egress}
    \textbf{Contrôle du trafic sortant des pods}

    \vspace{0.3cm}

    \begin{exemple}[Cas d'usage]
        \begin{itemize}
            \item Créer une whitelist interne
            \item Bloquer l'accès Internet
            \item Micro-segmentation réseau
        \end{itemize}
    \end{exemple}

    \vspace{0.3cm}

    \begin{important}[Prérequis]
        Nécessite un CNI compatible (Calico, Cilium, Weave Net)
    \end{important}
\end{frame}

\begin{frame}{Solutions avancées (hors scope BUT)}
    \textbf{Service Mesh (Istio, Linkerd)}

    \begin{itemize}
        \item Chiffrement mTLS automatique
        \item Telemetry et tracing distribué
        \item Traffic management (retry, circuit breaker)
        \item Canary deployments
    \end{itemize}

    \vspace{0.5cm}

    \begin{block}{Note}
        Ces solutions dépassent le niveau BUT mais sont utilisées en production.
    \end{block}
\end{frame}

% ================================
% VOLUMES
% ================================
\section{Volumes - Persistance}

\begin{frame}[plain,c]
    \centering
    \Huge\textcolor{iutblue}{Partie 3}

    \vspace{0.5cm}

    \huge Volumes et Persistance

    \vspace{1cm}

    \large\textcolor{iutgray}{Stockage de données dans Kubernetes}
\end{frame}

\begin{frame}{Problématique}
    \begin{important}[Concept fondamental]
        Les conteneurs sont \textbf{éphémères} : quand un pod est détruit, toutes ses données sont perdues.
    \end{important}

    \vspace{0.5cm}

    \textbf{Solution :} Les volumes permettent de conserver les données au-delà du cycle de vie du pod.
\end{frame}

\begin{frame}{Déclaration vs Montage}
    \textbf{Deux étapes distinctes :}

    \begin{enumerate}
        \item \textcolor{iutblue}{\textbf{Déclaration}} du volume dans \texttt{spec.volumes} (niveau pod)
        \item \textcolor{iutgreen}{\textbf{Montage}} du volume dans \texttt{volumeMounts} (niveau conteneur)
    \end{enumerate}

    \vspace{0.5cm}

    \begin{block}{Analogie}
        \begin{itemize}
            \item Déclaration = Créer le disque dur
            \item Montage = Brancher le disque dans l'ordinateur
        \end{itemize}
    \end{block}
\end{frame}

\begin{frame}[fragile]{Exemple - Déclaration et montage}
\begin{lstlisting}[language=yaml]
apiVersion: v1
kind: Pod
metadata:
  name: webapp
spec:
  # 1. DECLARATION (niveau pod)
  volumes:
    - name: config-volume
      configMap:
        name: app-config

  containers:
    - name: webapp
      image: nginx
      # 2. MONTAGE (niveau conteneur)
      volumeMounts:
        - name: config-volume
          mountPath: /etc/config
          readOnly: true
\end{lstlisting}
\end{frame}

\begin{frame}{Volumes non persistants (éphémères)}
    \begin{columns}
        \begin{column}{0.5\textwidth}
            \textbf{emptyDir}
            \begin{itemize}
                \item Créé avec le pod
                \item Supprimé avec le pod
                \item Partagé entre conteneurs
            \end{itemize}

            \textbf{Cas d'usage :}
            \begin{itemize}
                \item Cache temporaire
                \item Fichiers de travail
            \end{itemize}
        \end{column}

        \begin{column}{0.5\textwidth}
            \textbf{ConfigMap / Secret}
            \begin{itemize}
                \item Lecture seule
                \item Lié au pod
                \item Configuration/credentials
            \end{itemize}

            \textbf{Cas d'usage :}
            \begin{itemize}
                \item Fichiers de config
                \item Variables d'env
                \item Certificats
            \end{itemize}
        \end{column}
    \end{columns}
\end{frame}

\begin{frame}{Shared Volume dans un Pod}
    \begin{center}
        \includegraphics[height=0.65\textheight]{../../images/Shared_Volume_in_Pod2023.png}
    \end{center}
    \begin{block}{Concept}
        Plusieurs conteneurs dans un même pod peuvent partager un volume.
    \end{block}
\end{frame}

\begin{frame}{Volumes persistants - Architecture PV/PVC}
    \begin{center}
        \begin{tikzpicture}[
            node distance=2cm,
            box/.style={rectangle, draw, fill=blue!10, text width=2.5cm, text centered, minimum height=1cm, font=\small},
            arrow/.style={->, >=stealth, thick}
        ]
            \node[box, fill=orange!20] (admin) {Administrateur};
            \node[box, fill=cyan!20, below=of admin] (pv) {PersistentVolume\\(stockage réel)};
            \node[box, fill=green!20, right=2.5cm of pv] (pvc) {PersistentVolumeClaim\\(demande)};
            \node[box, fill=blue!20, above=of pvc] (dev) {Développeur};
            \node[box, fill=red!20, below=of pvc] (pod) {Pod};

            \draw[arrow] (admin) -- node[left, font=\tiny] {provisionne} (pv);
            \draw[arrow] (dev) -- node[right, font=\tiny] {crée} (pvc);
            \draw[arrow] (pvc) -- node[above, font=\tiny] {binding} (pv);
            \draw[arrow] (pod) -- node[right, font=\tiny] {utilise} (pvc);
        \end{tikzpicture}
    \end{center}

    \textbf{Séparation des responsabilités}
\end{frame}

\begin{frame}{PersistentVolume - Access Modes}
    \begin{itemize}
        \item \texttt{ReadWriteOnce (RWO)} : Lecture/écriture par un seul nœud
        \item \texttt{ReadOnlyMany (ROX)} : Lecture seule par plusieurs nœuds
        \item \texttt{ReadWriteMany (RWX)} : Lecture/écriture par plusieurs nœuds
    \end{itemize}

    \vspace{0.5cm}

    \textbf{Reclaim Policy :}
    \begin{itemize}
        \item \texttt{Retain} : Données conservées après suppression PVC
        \item \texttt{Delete} : Volume supprimé automatiquement
        \item \texttt{Recycle} : Données effacées, volume recyclé (déprécié)
    \end{itemize}
\end{frame}

\begin{frame}{PersistentVolume - Illustration}
    \begin{center}
        \includegraphics[width=0.7\textwidth]{../../images/PersistentVolume2023.png}
    \end{center}
\end{frame}

\begin{frame}{PersistentVolumeClaim}
    \begin{center}
        \includegraphics[width=0.75\textwidth]{../../images/1._PersistentVolumeClaim.png}
    \end{center}
\end{frame}

\begin{frame}{PVC utilisé dans un Pod}
    \begin{center}
        \includegraphics[width=0.75\textwidth]{../../images/2._PersistentVolumeClaim_Used_In_a_Pod.png}
    \end{center}
\end{frame}

\begin{frame}{Tableau comparatif}
    \begin{center}
        \tiny
        \begin{tabular}{|l|l|l|c|}
            \hline
            \textbf{Type} & \textbf{Cycle de vie} & \textbf{Cas d'usage} & \textbf{Persistant ?} \\ \hline
            \texttt{emptyDir} & Lié au pod & Cache temporaire & Non \\ \hline
            \texttt{configMap} & Lié au ConfigMap & Configuration & Non \\ \hline
            \texttt{secret} & Lié au Secret & Credentials & Non \\ \hline
            \texttt{PV/PVC} & Découplé du pod & Bases de données & \textcolor{iutgreen}{\textbf{Oui}} \\ \hline
        \end{tabular}
    \end{center}

    \vspace{0.5cm}

    \begin{block}{Règle simple}
        \begin{itemize}
            \item \textbf{Stateless} (frontend) : Pas de PVC
            \item \textbf{Stateful} (database) : PVC obligatoire
        \end{itemize}
    \end{block}
\end{frame}

% ================================
% DÉMONSTRATION
% ================================
\section{Démonstration}

\begin{frame}[plain,c]
    \centering
    \Huge\textcolor{iutblue}{Démonstration}

    \vspace{0.5cm}

    \huge Application front + back

    \vspace{1cm}

    \large\textcolor{iutgray}{Mise en pratique des concepts}
\end{frame}

\begin{frame}{Application front + back (chapitre 3)}
    \textbf{Concepts illustrés :}

    \begin{enumerate}
        \item \textbf{Load balancing} : Plusieurs réplicas backend
        \item \textbf{High Availability} : Si un pod tombe, Service redirige
        \item \textbf{Health checks} : Liveness \& Readiness probes
        \item \textbf{Ingress} : Routage basé sur path
    \end{enumerate}

    \vspace{0.5cm}

    \begin{exemple}[Démo live]
        \begin{itemize}
            \item Déployer backend (3 réplicas) + frontend (2 réplicas)
            \item Tester load balancing en tuant des pods
            \item Observer récupération automatique
        \end{itemize}
    \end{exemple}
\end{frame}

\begin{frame}[fragile]{Health Checks - Probes}
\begin{lstlisting}[language=yaml]
containers:
  - name: api
    image: backend:v1
    livenessProbe:
      httpGet:
        path: /health
        port: 3000
      initialDelaySeconds: 10
      periodSeconds: 5
    readinessProbe:
      httpGet:
        path: /ready
        port: 3000
      initialDelaySeconds: 5
\end{lstlisting}

    \textbf{Différence :}
    \begin{itemize}
        \item \textbf{Liveness} : Échec $\rightarrow$ redémarrage pod
        \item \textbf{Readiness} : Échec $\rightarrow$ retrait des endpoints
    \end{itemize}
\end{frame}

% ================================
% SYNTHÈSE
% ================================
\section{Synthèse}

\begin{frame}[plain,c]
    \centering
    \Huge\textcolor{iutblue}{Synthèse}

    \vspace{0.5cm}

    \huge Récapitulatif

    \vspace{1cm}

    \large\textcolor{iutgray}{Les points clés à retenir}
\end{frame}

\begin{frame}{Points clés - Labels vs Annotations}
    \begin{itemize}
        \item \textbf{Labels} : Sélection (selectors) $\rightarrow$ comportement K8s
        \item \textbf{Annotations} : Métadonnées techniques (sans sélection)
        \item Nomenclature cohérente = éviter collisions
    \end{itemize}

    \vspace{0.5cm}

    \begin{block}{À retenir}
        Labels influencent le comportement, annotations NON.
    \end{block}
\end{frame}

\begin{frame}{Points clés - Ingress / Services / Pods}
    \begin{itemize}
        \item \textbf{Ingress (objet)} : Règles de routage
        \item \textbf{Ingress Controller} : Implémentation (nginx, traefik...)
        \item \textbf{Service} : Abstraction ciblant pods via labels
        \item \textbf{NetworkPolicy Egress} : Contrôle trafic sortant
    \end{itemize}

    \vspace{0.5cm}

    \begin{important}[Critique]
        Ingress objet $\neq$ Ingress Controller. L'un ne fonctionne pas sans l'autre !
    \end{important}
\end{frame}

\begin{frame}{Points clés - Volumes}
    \begin{itemize}
        \item \textbf{Déclaration} : \texttt{spec.volumes} (niveau pod)
        \item \textbf{Montage} : \texttt{volumeMounts} (niveau conteneur)
        \item \textbf{Non persistants} : emptyDir, ConfigMap, Secret
        \item \textbf{Persistants} : PV/PVC (découplés du pod)
    \end{itemize}

    \vspace{0.5cm}

    \begin{block}{Règle}
        \begin{itemize}
            \item Stateless : Pas de PVC
            \item Stateful (BDD) : PVC obligatoire
        \end{itemize}
    \end{block}
\end{frame}

\begin{frame}{Questions pour auto-évaluation}
    \begin{enumerate}
        \item Quelle est la différence entre un label et une annotation ?
        \item Que se passe-t-il si vous créez un Ingress sans Ingress Controller ?
        \item Expliquez le flux HTTP de l'utilisateur jusqu'au pod.
        \item Différence entre \texttt{spec.volumes} et \texttt{volumeMounts} ?
        \item Pourquoi utiliser PVC au lieu d'emptyDir pour une BDD ?
    \end{enumerate}
\end{frame}

% ================================
% FIN
% ================================
\begin{frame}[plain,c]
    \centering
    \Huge\textcolor{iutblue}{Questions ?}

    \vspace{1.5cm}

    \large
    \textcolor{iutgray}{Merci de votre attention}

    \vspace{0.5cm}

    \normalsize
    \textcolor{iutcyan}{Prochaines étapes : Pratiquer avec chapitre 3 GitLab}
\end{frame}

\begin{frame}[plain,c]
    \centering
    \Huge\textcolor{iutblue}{Crédits et Ressources}

    \vspace{1cm}

    \Large\textbf{Linux Foundation}

    \vspace{0.5cm}

    \normalsize
    \textbf{Sources des illustrations et du contenu :}

    \vspace{0.3cm}

    \begin{itemize}
        \item \textbf{Linux Foundation} - Introduction to Kubernetes (LFS158x)
        \item Cours officiel : \url{https://training.linuxfoundation.org/training/introduction-to-kubernetes/}
        \item Documentation officielle Kubernetes : \url{https://kubernetes.io/docs/}
    \end{itemize}

    \vspace{0.7cm}

    \small
    \textcolor{iutgray}{Les schémas et illustrations proviennent du cours officiel de la Linux Foundation.\\
    Licence Creative Commons - Usage éducatif.}
\end{frame}

\end{document}
