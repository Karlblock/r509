\documentclass[11pt,a4paper]{article}
\usepackage[utf8]{inputenc}
\usepackage[french]{babel}
\usepackage[T1]{fontenc}
\usepackage{geometry}
\usepackage{array}
\usepackage{graphicx}
\usepackage{amssymb}
\usepackage{multicol}
\usepackage{fancyhdr}

\geometry{top=2cm, bottom=2cm, left=2cm, right=2cm}
\pagestyle{fancy}
\fancyhf{}
\fancyhead[L]{QCM -- TD3 Helm}
\fancyhead[R]{Page \thepage/4}
\fancyfoot[C]{ML 2024}

\begin{document}

\begin{center}
\textbf{\Large BUT Informatique : QCM Helm -- TD3}
\end{center}

\vspace{0.3cm}

\noindent
\begin{tabular}{|p{0.6\textwidth}|p{0.35\textwidth}|}
\hline
\textbf{Nom et prénom :} & \\
\dotfill & \\
 & \\
\hline
\end{tabular}

\vspace{0.5cm}

\noindent
\textbf{Contrôle de compréhension -- TD3 : Helm}

\vspace{0.3cm}

\noindent
Rappel : Ce questionnaire est un QCM de validation de la compréhension du TD3 sur Helm.
\textbf{Calculatrice non autorisée -- Aucun document n'est autorisé !}

\vspace{0.2cm}

\noindent
Les questions ont nécessairement \textbf{une bonne réponse et une seule}, sauf si elles font apparaître le symbole \textbf{$\clubsuit$} et peuvent alors avoir \textbf{plusieurs bonnes réponses}. Une bonne réponse vaut \textbf{2 points}, une mauvaise réponse vaut \textbf{-1 point}, une absence de réponse vaut \textbf{0 point}.

\vspace{0.5cm}

\hrule

\vspace{0.5cm}

\section*{QCM :}

\subsection*{Question 1 : Qu'est-ce qu'un Chart Helm ?}
\begin{itemize}
    \item[$\square$] Un graphique de monitoring des ressources Kubernetes
    \item[$\square$] Un package contenant tous les fichiers nécessaires pour déployer une application Kubernetes
    \item[$\square$] Un outil de visualisation de cluster
    \item[$\square$] Un type de service Kubernetes
\end{itemize}

\subsection*{Question 2 : Comment appelle-t-on le déploiement d'un Chart dans un cluster Kubernetes ?}
\begin{itemize}
    \item[$\square$] Un deployment
    \item[$\square$] Une instance
    \item[$\square$] Une release
    \item[$\square$] Un package
\end{itemize}

\subsection*{Question 3 : Quelle est la structure typique d'un Chart Helm ?}
\begin{itemize}
    \item[$\square$] Dockerfile, docker-compose.yml, README.md
    \item[$\square$] Chart.yaml, templates/, values.yaml, charts/
    \item[$\square$] deployment.yaml, service.yaml, ingress.yaml
    \item[$\square$] package.json, node\_modules/, dist/
\end{itemize}

\subsection*{Question 4 : À quoi sert le fichier \texttt{Chart.yaml} ?}
\begin{itemize}
    \item[$\square$] À définir les valeurs des variables du Chart
    \item[$\square$] À contenir les manifestes Kubernetes
    \item[$\square$] À définir le nom, la version, la description et les dépendances du Chart
    \item[$\square$] À stocker les secrets de l'application
\end{itemize}

\subsection*{Question 5 : Quel est le rôle du dossier \texttt{templates/} dans un Chart ?}
\begin{itemize}
    \item[$\square$] Stocker les Charts dépendants
    \item[$\square$] Contenir les modèles de manifestes Kubernetes au format YAML
    \item[$\square$] Définir les versions du Chart
    \item[$\square$] Gérer les logs de déploiement
\end{itemize}

\subsection*{Question 6 : À quoi sert le fichier \texttt{values.yaml} ?}
\begin{itemize}
    \item[$\square$] À définir les secrets de l'application
    \item[$\square$] À lister toutes les releases déployées
    \item[$\square$] À contenir les valeurs par défaut pour les variables utilisées dans les templates
    \item[$\square$] À spécifier les dépendances du Chart
\end{itemize}

\subsection*{Question 7 : Quelle syntaxe permet de référencer une variable dans un template Helm ?}
\begin{itemize}
    \item[$\square$] \texttt{\$\{.Values.ma.variable\}}
    \item[$\square$] \texttt{\{\{ .Values.ma.variable \}\}}
    \item[$\square$] \texttt{@Values.ma.variable@}
    \item[$\square$] \texttt{<\%= Values.ma.variable \%>}
\end{itemize}

\subsection*{Question 8 : Quelle commande permet d'installer ou mettre à jour un Chart dans un namespace ?}
\begin{itemize}
    \item[$\square$] \texttt{helm deploy <name> <chart> --namespace <ns>}
    \item[$\square$] \texttt{helm install <name> <chart> --namespace <ns>}
    \item[$\square$] \texttt{helm upgrade <name> <chart> --namespace <ns>}
    \item[$\square$] \texttt{helm apply <name> <chart> --namespace <ns>}
\end{itemize}

\subsection*{Question 9 : Comment packager un Chart Helm ?}
\begin{itemize}
    \item[$\square$] \texttt{helm build <chart-path>}
    \item[$\square$] \texttt{helm package <chart-path>}
    \item[$\square$] \texttt{helm create <chart-path>}
    \item[$\square$] \texttt{helm zip <chart-path>}
\end{itemize}

\subsection*{Question 10 : Où sont stockés les Charts dépendants dans l'arborescence d'un Chart ?}
\begin{itemize}
    \item[$\square$] Dans le dossier \texttt{dependencies/}
    \item[$\square$] Dans le dossier \texttt{charts/}
    \item[$\square$] Dans le dossier \texttt{templates/}
    \item[$\square$] Dans le fichier \texttt{Chart.yaml}
\end{itemize}

\subsection*{Question 11 : Comment spécifier une dépendance dans un Chart Helm ?}
\begin{itemize}
    \item[$\square$] En ajoutant une section \texttt{dependencies:} dans \texttt{Chart.yaml}
    \item[$\square$] En créant un fichier \texttt{dependencies.yaml}
    \item[$\square$] En utilisant la commande \texttt{helm add dependency}
    \item[$\square$] En ajoutant le Chart dans le dossier \texttt{node\_modules/}
\end{itemize}

\subsection*{Question 12 : Quel fichier permet de définir des fonctions réutilisables dans un Chart ?}
\begin{itemize}
    \item[$\square$] \texttt{functions.yaml}
    \item[$\square$] \texttt{\_helpers.tpl}
    \item[$\square$] \texttt{utils.yaml}
    \item[$\square$] \texttt{common.tpl}
\end{itemize}

\subsection*{Question 13 $\clubsuit$ : Quelles commandes permettent de surcharger des valeurs lors du déploiement d'un Chart ?}
\begin{itemize}
    \item[$\square$] \texttt{helm upgrade <name> -f values.yaml}
    \item[$\square$] \texttt{helm upgrade <name> --set cle.variable='valeur'}
    \item[$\square$] \texttt{helm upgrade <name> --values custom.yaml}
    \item[$\square$] \texttt{helm upgrade <name> --override cle.variable='valeur'}
\end{itemize}

\subsection*{Question 14 : Quelle commande permet de voir l'historique des releases d'un Chart ?}
\begin{itemize}
    \item[$\square$] \texttt{helm logs <name>}
    \item[$\square$] \texttt{helm history <name>}
    \item[$\square$] \texttt{helm status <name>}
    \item[$\square$] \texttt{helm list <name>}
\end{itemize}

\subsection*{Question 15 $\clubsuit$ : Quels sont les avantages de Helm pour Kubernetes ?}
\begin{itemize}
    \item[$\square$] Gestion reproductible des déploiements
    \item[$\square$] Paramétrage des manifestes sans modification directe
    \item[$\square$] Remplacement complet de kubectl
    \item[$\square$] Infrastructure as Code (IaC)
\end{itemize}

\newpage

\section*{Question ouverte :}

\subsection*{Question 16 : Expliquez la différence entre un Chart et une Release dans Helm. Donnez un exemple concret d'utilisation.}

\vspace{0.3cm}

\noindent
Rédigez ci-dessous \hfill $\square$ 0\% \quad $\square$ 25\% \quad $\square$ 50\% \quad $\square$ 75\% \quad $\square$ 100\%

\vspace{0.3cm}

\noindent
\dotfill \\
\dotfill \\
\dotfill \\
\dotfill \\
\dotfill \\
\dotfill \\
\dotfill \\
\dotfill \\
\dotfill \\
\dotfill \\

\vspace{1cm}

\subsection*{Question 17 : Pourquoi est-il important de versionner ses Charts Helm ? Donnez deux raisons concrètes.}

\vspace{0.3cm}

\noindent
Rédigez ci-dessous \hfill $\square$ 0\% \quad $\square$ 25\% \quad $\square$ 50\% \quad $\square$ 75\% \quad $\square$ 100\%

\vspace{0.3cm}

\noindent
\dotfill \\
\dotfill \\
\dotfill \\
\dotfill \\
\dotfill \\
\dotfill \\
\dotfill \\
\dotfill \\

\newpage

\section*{Mise en situation :}

\subsection*{Contexte : Déploiement multi-environnements}

Vous êtes DevOps dans une entreprise qui développe une application web. Vous devez déployer cette application dans trois environnements différents :
\begin{itemize}
    \item \textbf{Développement} : 1 replica, 512 MB RAM, pas de ressources limitées
    \item \textbf{Staging} : 2 replicas, 1 GB RAM, limites de ressources activées
    \item \textbf{Production} : 5 replicas, 2 GB RAM, limites strictes, monitoring activé
\end{itemize}

L'application utilise une base de données PostgreSQL qui doit également être déployée avec des configurations différentes selon l'environnement.

\vspace{0.5cm}

\subsection*{Question 18 : Comment utiliseriez-vous Helm pour gérer ces trois environnements différents avec un seul Chart ? Décrivez l'organisation de votre Chart et expliquez comment vous géreriez les différentes configurations.}

\vspace{0.3cm}

\noindent
Rédigez ci-dessous \hfill $\square$ 0\% \quad $\square$ 25\% \quad $\square$ 50\% \quad $\square$ 75\% \quad $\square$ 100\%

\vspace{0.3cm}

\noindent
\dotfill \\
\dotfill \\
\dotfill \\
\dotfill \\
\dotfill \\
\dotfill \\
\dotfill \\
\dotfill \\
\dotfill \\
\dotfill \\
\dotfill \\
\dotfill \\
\dotfill \\
\dotfill \\

\newpage

\subsection*{Question 19 : Dans le contexte de la question 18, expliquez comment vous organiseriez les dépendances entre votre Chart applicatif et le Chart PostgreSQL. Justifiez votre approche.}

\vspace{0.3cm}

\noindent
Rédigez ci-dessous \hfill $\square$ 0\% \quad $\square$ 25\% \quad $\square$ 50\% \quad $\square$ 75\% \quad $\square$ 100\%

\vspace{0.3cm}

\noindent
\dotfill \\
\dotfill \\
\dotfill \\
\dotfill \\
\dotfill \\
\dotfill \\
\dotfill \\
\dotfill \\
\dotfill \\
\dotfill \\

\vspace{1cm}

\subsection*{Question 20 : Vous devez intégrer ce Chart Helm dans un pipeline CI/CD GitLab. Décrivez les étapes principales que vous mettriez en place, de la modification du code jusqu'au déploiement en production.}

\vspace{0.3cm}

\noindent
Rédigez ci-dessous \hfill $\square$ 0\% \quad $\square$ 25\% \quad $\square$ 50\% \quad $\square$ 75\% \quad $\square$ 100\%

\vspace{0.3cm}

\noindent
\dotfill \\
\dotfill \\
\dotfill \\
\dotfill \\
\dotfill \\
\dotfill \\
\dotfill \\
\dotfill \\
\dotfill \\
\dotfill \\
\dotfill \\
\dotfill \\

\vspace{1cm}

\begin{center}
\textbf{Fin du QCM}
\end{center}

\end{document}
