\documentclass[11pt,a4paper]{article}
\usepackage[utf8]{inputenc}
\usepackage[french]{babel}
\usepackage[T1]{fontenc}
\usepackage{geometry}
\usepackage{listings}
\usepackage{xcolor}
\usepackage{graphicx}
\usepackage{fancyhdr}
\usepackage{hyperref}

\geometry{top=2cm, bottom=2cm, left=2cm, right=2cm}

% Configuration des listings
\lstset{
    basicstyle=\ttfamily\small,
    breaklines=true,
    frame=single,
    backgroundcolor=\color{gray!10},
    keywordstyle=\color{blue},
    commentstyle=\color{green!50!black},
    stringstyle=\color{red},
    showstringspaces=false,
    numbers=left,
    numberstyle=\tiny\color{gray},
    tabsize=2
}

\pagestyle{fancy}
\fancyhf{}
\fancyhead[L]{TP1 -- Kubernetes avec Kind}
\fancyhead[R]{R5.09 -- 2024/2025}
\fancyfoot[C]{\thepage}

\title{\textbf{TP1 : Installation et Configuration Kubernetes}\\
\large Cluster local avec Kind et premiers déploiements}
\author{IUT Grand Ouest Normandie -- BUT Informatique S5}
\date{Année 2024/2025}

\begin{document}

\maketitle

\tableofcontents

\newpage

\section{Prérequis et environnement}

\subsection{Configuration système recommandée}

\textbf{Machine physique} : Recommandé pour des performances optimales

\textbf{Ressources minimales} :
\begin{itemize}
    \item CPU : 4 cores
    \item RAM : 8 GB
    \item Disque : 40 GB disponibles
    \item OS : Linux (Ubuntu 22.04+, Debian 12+, Parrot OS)
\end{itemize}

\subsection{Configuration VM Proxmox (si nécessaire)}

\textbf{Type CPU} : Changer de \texttt{x86-64-v2-AES} à \texttt{host}

\begin{lstlisting}[language=bash]
# Sur l'hote Proxmox, verifier la virtualisation imbriquee
# Intel :
cat /sys/module/kvm_intel/parameters/nested

# AMD :
cat /sys/module/kvm_amd/parameters/nested

# Si retourne N ou 0, activer :
# Intel :
echo "options kvm_intel nested=1" > /etc/modprobe.d/kvm-intel.conf
modprobe -r kvm_intel
modprobe kvm_intel

# AMD :
echo "options kvm_amd nested=1" > /etc/modprobe.d/kvm-amd.conf
modprobe -r kvm_amd
modprobe kvm_amd
\end{lstlisting}

\textbf{Dans la VM, verifier les flags CPU} :

\begin{lstlisting}[language=bash]
grep -E 'vmx|svm' /proc/cpuinfo
ls -la /dev/kvm
\end{lstlisting}

\subsection{Vérification Docker}

\begin{lstlisting}[language=bash]
# Verifier Docker
docker --version
docker info | grep -E "Cgroup|Kernel"

# Doit afficher :
# Cgroup Driver: systemd
# Cgroup Version: 2
\end{lstlisting}

\section{Installation de Kind}

\subsection{Installation sur Linux}

\begin{lstlisting}[language=bash]
# Telecharger et installer Kind
curl -Lo ./kind https://kind.sigs.k8s.io/dl/v0.30.0/kind-linux-amd64
chmod +x ./kind
sudo mv ./kind /usr/local/bin/kind

# Verifier l'installation
kind version
\end{lstlisting}

\subsection{Installation de kubectl}

\begin{lstlisting}[language=bash]
# Telecharger kubectl
curl -LO "https://dl.k8s.io/release/$(curl -L -s https://dl.k8s.io/release/stable.txt)/bin/linux/amd64/kubectl"

# Installer
sudo install -o root -g root -m 0755 kubectl /usr/local/bin/kubectl

# Verifier
kubectl version --client
\end{lstlisting}

\section{Création du cluster Kind}

\subsection{Cluster simple (single-node)}

\begin{lstlisting}[language=bash]
# Creer un cluster basique
kind create cluster --name mon-cluster

# Verifier le cluster
kubectl cluster-info --context kind-mon-cluster
kubectl get nodes
\end{lstlisting}

\subsection{Cluster avec support Ingress}

Créer le fichier \texttt{kind-ingress-config.yaml} :

\begin{lstlisting}[language=yaml]
kind: Cluster
apiVersion: kind.x-k8s.io/v1alpha4
nodes:
- role: control-plane
  kubeadmConfigPatches:
  - |
    kind: InitConfiguration
    nodeRegistration:
      kubeletExtraArgs:
        node-labels: "ingress-ready=true"
  extraPortMappings:
  - containerPort: 80
    hostPort: 80
    protocol: TCP
  - containerPort: 443
    hostPort: 443
    protocol: TCP
\end{lstlisting}

\begin{lstlisting}[language=bash]
# Creer le cluster avec configuration
kind create cluster --config kind-ingress-config.yaml --name cluster-ingress

# Verifier
kubectl get nodes
docker ps | grep kind
\end{lstlisting}

\subsection{Commandes Kind essentielles}

\begin{lstlisting}[language=bash]
# Lister les clusters
kind get clusters

# Obtenir le kubeconfig
kind get kubeconfig --name cluster-ingress

# Charger une image Docker dans le cluster
docker pull nginx:latest
kind load docker-image nginx:latest --name cluster-ingress

# Supprimer un cluster
kind delete cluster --name cluster-ingress
\end{lstlisting}

\section{Installation Ingress NGINX}

\subsection{Déploiement du contrôleur}

\begin{lstlisting}[language=bash]
# Installer ingress-nginx
kubectl apply -f https://raw.githubusercontent.com/kubernetes/ingress-nginx/main/deploy/static/provider/kind/deploy.yaml

# Attendre que le controleur soit pret
kubectl wait --namespace ingress-nginx \
  --for=condition=ready pod \
  --selector=app.kubernetes.io/component=controller \
  --timeout=90s

# Verifier
kubectl get pods -n ingress-nginx
kubectl get svc -n ingress-nginx
\end{lstlisting}

\subsection{Troubleshooting Ingress}

\textbf{Problème : Webhook validation timeout}

\begin{lstlisting}[language=bash]
# Verifier l'etat
kubectl get pods -n ingress-nginx

# Si jobs admission en erreur, creer le secret manuellement
kubectl create secret generic ingress-nginx-admission \
  --from-literal=cert=dummy \
  --from-literal=key=dummy \
  -n ingress-nginx

# Supprimer le webhook (environnement de test)
kubectl delete validatingwebhookconfiguration ingress-nginx-admission

# Redemarrer le controleur
kubectl rollout restart deployment/ingress-nginx-controller -n ingress-nginx

# Attendre
kubectl wait --namespace ingress-nginx \
  --for=condition=ready pod \
  --selector=app.kubernetes.io/component=controller \
  --timeout=180s
\end{lstlisting}

\section{Premier déploiement : nginx}

\subsection{Manifeste Deployment}

Créer \texttt{nginx-deployment.yaml} :

\begin{lstlisting}[language=yaml]
apiVersion: apps/v1
kind: Deployment
metadata:
  name: nginx-deployment
  labels:
    app: nginx
spec:
  replicas: 2
  selector:
    matchLabels:
      app: nginx
  template:
    metadata:
      labels:
        app: nginx
    spec:
      containers:
      - name: nginx
        image: nginx:1.27
        ports:
        - containerPort: 80
\end{lstlisting}

\subsection{Manifeste Service}

Créer \texttt{nginx-service.yaml} :

\begin{lstlisting}[language=yaml]
apiVersion: v1
kind: Service
metadata:
  name: nginx-service
spec:
  selector:
    app: nginx
  ports:
  - protocol: TCP
    port: 80
    targetPort: 80
  type: ClusterIP
\end{lstlisting}

\subsection{Manifeste Ingress}

Créer \texttt{nginx-ingress.yaml} :

\begin{lstlisting}[language=yaml]
apiVersion: networking.k8s.io/v1
kind: Ingress
metadata:
  name: nginx-ingress
  annotations:
    nginx.ingress.kubernetes.io/rewrite-target: /
spec:
  ingressClassName: nginx
  rules:
  - host: nginx.local
    http:
      paths:
      - path: /
        pathType: Prefix
        backend:
          service:
            name: nginx-service
            port:
              number: 80
\end{lstlisting}

\subsection{Déploiement et test}

\begin{lstlisting}[language=bash]
# Appliquer les manifestes
kubectl apply -f nginx-deployment.yaml
kubectl apply -f nginx-service.yaml
kubectl apply -f nginx-ingress.yaml

# Verifier
kubectl get pods -l app=nginx
kubectl get svc nginx-service
kubectl get ingress nginx-ingress

# Ajouter l'entree DNS
echo "127.0.0.1 nginx.local" | sudo tee -a /etc/hosts

# Tester
curl http://nginx.local
\end{lstlisting}

\section{Accès via IP LAN}

\subsection{Configuration réseau}

\begin{lstlisting}[language=bash]
# Trouver votre IP LAN
hostname -I | awk '{print $1}'

# Verifier que kind ecoute sur toutes les interfaces
sudo ss -tulpn | grep :80
# Devrait afficher : 0.0.0.0:80
\end{lstlisting}

\subsection{Accès depuis d'autres machines}

Sur les machines clientes (par exemple IP serveur : 192.168.56.11) :

\begin{lstlisting}[language=bash]
# Ajouter l'entree DNS
echo "192.168.56.11 nginx.local" | sudo tee -a /etc/hosts

# Tester
curl http://nginx.local

# Ou avec header Host
curl -H "Host: nginx.local" http://192.168.56.11
\end{lstlisting}

\section{Commandes kubectl essentielles}

\subsection{Gestion des ressources}

\begin{lstlisting}[language=bash]
# Lister les pods
kubectl get pods
kubectl get pods -o wide
kubectl get pods --all-namespaces

# Lister les deployments
kubectl get deployments

# Lister les services
kubectl get services

# Lister les ingress
kubectl get ingress

# Tout voir
kubectl get all
kubectl get all -A
\end{lstlisting}

\subsection{Informations détaillées}

\begin{lstlisting}[language=bash]
# Describe (details complets)
kubectl describe pod <pod-name>
kubectl describe deployment <deployment-name>
kubectl describe service <service-name>

# Logs
kubectl logs <pod-name>
kubectl logs <pod-name> -f  # follow
kubectl logs deployment/<deployment-name>

# Events
kubectl get events --sort-by='.lastTimestamp'
\end{lstlisting}

\subsection{Manipulation des ressources}

\begin{lstlisting}[language=bash]
# Appliquer un manifest
kubectl apply -f fichier.yaml
kubectl apply -f repertoire/

# Supprimer des ressources
kubectl delete -f fichier.yaml
kubectl delete pod <pod-name>
kubectl delete deployment <deployment-name>

# Modifier une ressource
kubectl edit deployment <deployment-name>

# Scaler un deployment
kubectl scale deployment <deployment-name> --replicas=3
\end{lstlisting}

\subsection{Exec et port-forward}

\begin{lstlisting}[language=bash]
# Executer une commande dans un pod
kubectl exec <pod-name> -- ls /
kubectl exec -it <pod-name> -- /bin/bash

# Port-forward
kubectl port-forward pod/<pod-name> 8080:80
kubectl port-forward service/<service-name> 8080:80
\end{lstlisting}

\section{Contextes et namespaces}

\subsection{Gestion des contextes}

\begin{lstlisting}[language=bash]
# Lister les contextes
kubectl config get-contexts

# Changer de contexte
kubectl config use-context kind-cluster-ingress

# Voir le contexte actuel
kubectl config current-context
\end{lstlisting}

\subsection{Gestion des namespaces}

\begin{lstlisting}[language=bash]
# Lister les namespaces
kubectl get namespaces

# Creer un namespace
kubectl create namespace dev

# Definir un namespace par defaut
kubectl config set-context --current --namespace=dev

# Deployer dans un namespace specifique
kubectl apply -f fichier.yaml -n dev
kubectl get pods -n dev
\end{lstlisting}

\section{Nettoyage}

\begin{lstlisting}[language=bash]
# Supprimer les ressources
kubectl delete -f nginx-ingress.yaml
kubectl delete -f nginx-service.yaml
kubectl delete -f nginx-deployment.yaml

# Supprimer le cluster
kind delete cluster --name cluster-ingress

# Retirer l'entree /etc/hosts
sudo sed -i '/nginx.local/d' /etc/hosts
\end{lstlisting}

\section{Troubleshooting}

\subsection{Problèmes courants}

\textbf{ImagePullBackOff} :
\begin{lstlisting}[language=bash]
# Charger l'image dans Kind
docker pull nginx:1.27
kind load docker-image nginx:1.27 --name cluster-ingress
\end{lstlisting}

\textbf{CrashLoopBackOff} :
\begin{lstlisting}[language=bash]
# Voir les logs
kubectl logs <pod-name>
kubectl logs <pod-name> --previous

# Voir les events
kubectl describe pod <pod-name>
\end{lstlisting}

\textbf{Pods Pending} :
\begin{lstlisting}[language=bash]
# Verifier les ressources des nodes
kubectl describe nodes

# Verifier les events
kubectl get events --sort-by='.lastTimestamp'
\end{lstlisting}

\textbf{Service ne répond pas} :
\begin{lstlisting}[language=bash]
# Verifier les endpoints
kubectl get endpoints <service-name>

# Verifier les labels
kubectl get pods --show-labels
kubectl describe service <service-name>
\end{lstlisting}

\section{Cheat Sheet rapide}

\begin{lstlisting}[language=bash]
# CLUSTER
kind create cluster --name mon-cluster
kind delete cluster --name mon-cluster
kind get clusters

# DEPLOIEMENT
kubectl apply -f fichier.yaml
kubectl delete -f fichier.yaml

# VISUALISATION
kubectl get pods
kubectl get all
kubectl describe pod <nom>
kubectl logs <pod-name>

# MANIPULATION
kubectl scale deployment <nom> --replicas=3
kubectl exec -it <pod> -- /bin/bash
kubectl port-forward svc/<service> 8080:80

# DEBUG
kubectl get events --sort-by='.lastTimestamp'
kubectl logs <pod> --previous
kubectl describe pod <pod>

# CONTEXTE
kubectl config get-contexts
kubectl config use-context <context>
kubectl config set-context --current --namespace=<ns>
\end{lstlisting}

\section{Ressources utiles}

\begin{itemize}
    \item Documentation Kind : \url{https://kind.sigs.k8s.io/}
    \item Documentation Kubernetes : \url{https://kubernetes.io/docs/}
    \item Ingress NGINX : \url{https://kubernetes.github.io/ingress-nginx/}
    \item kubectl Cheat Sheet : \url{https://kubernetes.io/docs/reference/kubectl/cheatsheet/}
\end{itemize}

\end{document}
